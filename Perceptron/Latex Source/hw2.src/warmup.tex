\section{Warm up: Feature expansion}

[10 points] Consider the concept class $C$ consisting of functions
function $f_r$ defined by a radius $r$ as follows:

\begin{equation}
f_r(x_1, x_2) = \left\{
    \begin{array}{rl}
      +1 & 4x_1^4 + 16x_2^4 \leq r;\\
      -1 & \mbox{otherwise}
    \end{array}
\right.
\label{eq:f_r}
\end{equation}

Clearly the hypothesis class is {\em not} linearly separable in
$\Re^2$. 

Construct a function $\phi(x_1, x_2)$ that maps examples to a new
space, such that the positive and negative examples are linearly
separable in that space. The answer to this question should consist of
two parts: (1) a function $\phi$ that maps the instances to the new
space, and (2) a proof that in the new space, the positive and
negative points are linearly separated. You can show this by producing
such a hyperplane in the new space (i.e. find a weight vector $\bw$
and a bias $b$ such that $\bw^T\phi(x_1, x_2) \geq b$ if, and only if,
$f_r(x_1, x_2) = +1$.

Hint: The feature transformation $\phi$ should not depend on $r$.



%%% Local Variables:
%%% mode: latex
%%% TeX-master: "hw"
%%% End:
